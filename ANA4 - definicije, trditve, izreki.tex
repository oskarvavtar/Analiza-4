\documentclass[11pt]{article}
\usepackage[utf8]{inputenc}
\usepackage[slovene]{babel}

\usepackage{amsthm}
\usepackage{amsmath, amssymb, amsfonts}
\usepackage{relsize}
\usepackage{mathrsfs}
\usepackage{bm}
\usepackage{calligra}

\newcommand{\R}{\mathbb{R}}
\newcommand{\N}{\mathbb{N}}
\newcommand{\dif}{\Delta}
\renewcommand{\b}{\mathbf{b}}
\newcommand{\f}{\mathbf{f}}
\newcommand{\g}{\mathbf{g}}
\newcommand{\x}{\mathbf{x}}
\newcommand{\y}{\mathbf{y}}
\newcommand{\z}{\mathbf{z}}
\newcommand{\uu}{\mathbf{u}}
\newcommand{\vv}{\mathbf{v}}
\newcommand{\w}{\mathbf{w}}
\newcommand{\A}{\mathbf{A}}
\newcommand{\C}{\mathcal{C}}
\newcommand{\glnr}{\text{GL}_n(\R)}
\newcommand{\set}[1]{\{#1\}}
\newcommand{\re}{\mathfrak{Re}}
\newcommand{\Lin}{\text{\calligra Lin}}
\newcommand{\0}{\mathbf{0}}

\theoremstyle{definition}
\newtheorem{definicija}{Definicija}[section]

\theoremstyle{definition}
\newtheorem{trditev}{Trditev}[section]

\theoremstyle{definition}
\newtheorem{izrek}{Izrek}[section]

\theoremstyle{definition}
\newtheorem{metoda}{Metoda}[section]

\newtheorem*{posledica}{Posledica}
\newtheorem*{opomba}{Opomba}
\newtheorem*{komentar}{Komentar}
\newtheorem{lema}{Lema}
\newtheorem*{dokaz}{Dokaz}
\newtheorem*{posplošitev}{Posplošitev}
\newtheorem*{dogovor}{Dogovor}
\newtheorem*{sklep}{Sklep}
\newtheorem{enacba}{Enačba}

\title{Analiza 4 - definicije, trditve in izreki}
\author{Oskar Vavtar \\
po predavanjih profesorice Jasne Prezelj}
\date{2021/22}

\begin{document}
\maketitle
\pagebreak
\tableofcontents
\pagebreak

% #################################################################################################

\section{Diferenčne enačbe}
\vspace{0.5cm}

% *************************************************************************************************

\subsection{Uvod}
\vspace{0.5cm}

\begin{definicija}[Diferenca]

Denimo, da je $y=f(t)$ dana funkcija.
\begin{itemize}
	\item 1. način: $\dif y_t ~=~ f(t+h) - f(t) ~=~ y_{t+h} - y_t$
	\item 2. način: $\dif y_t ~=~ f(t) - f(t-h) ~=~ y_t - y_{t-h}$
\end{itemize}
Posebej definiramo $\dif^0 y_t = y_t$. Velja
$$\dif^{n+1} y_t ~=~ \dif(\dif^n y_t)_t.$$

\end{definicija}
\vspace{0.5cm}

\begin{definicija}

\textit{Navadna diferenčna enačba} je enačba, ki vsebuje eno ali več diferenc,
$$F(t, \dif^0 y_t, \ldots, \dif^n y_t) ~=~ 0.$$
\textit{Red} diferenčne enačbe je red najvišje diference. Če je $F$ linearna v $\dif^k y_t$, je enačba linearna.

\end{definicija}
\vspace{0.5cm}

\begin{definicija}

\textit{Sistem $n$ diferenčnih enačb 1. reda} je dan z
\begin{align*}
y_1(t+1) ~&=~ f_1(t,y_1(t),\ldots,y_n(t)) \\
&\vdots \\
y_n(t+1) ~&=~ f_n(t,y_1(t),\ldots,y_n(t))
\end{align*} 
Če $t$ eksplicitno ne nastopa, rečemo, da je to \textit{avtonomen sistem}. Če so $f_1,\ldots,f_n$ linearne, lahko sistem zapišemo v matrični obliki:
\begin{align*}
\y ~&=~ \begin{bmatrix}
y_1 \\
\vdots \\
y_n
\end{bmatrix},~~~ f_i(t) ~=~ b_i(t) + a_{i1}y_1(t) + \ldots + a_{in}y_n(t), ~i = 1,\ldots, n\\ \\
\y(t+1) ~&=~ \begin{bmatrix}
a_{11}(t) & \ldots & a_{1n}(t) \\
\vdots & \ddots & \vdots \\
a_{n1}(t) & \ldots & a_{nn}(t)
\end{bmatrix} \begin{bmatrix}
y_1(t) \\
\vdots \\
y_n(t) 
\end{bmatrix} + \begin{bmatrix}
b_1(t) \\
\vdots \\
b_n(t)
\end{bmatrix} ~=~ \A(t)\y(t) + \b(t)
\end{align*}
Nelinearen sistem lahko vseeno zapišemo v vektorski obliki:
$$\f ~=~ \begin{bmatrix}
f_1 \\
\vdots \\
f_n
\end{bmatrix}, ~~~ \y(t+1) ~=~ \f(t, \y(t))$$

\end{definicija}
\vspace{0.5cm}

\begin{definicija}

Če je sistem oblike
$$\y_{m+1} = \A\y_m,$$
se imenuje \textit{homogen}.

\end{definicija}
\vspace{0.5cm}

% *************************************************************************************************

\subsection{Linearne diferenčne enačbe in \\sistemi linearnih diferenčnih enačb}
\vspace{0.5cm}

\begin{definicija}

\textit{Sistem linearnih diferencialnih enačb} reda $n$ je dan s predpisom
$$\y(t+1) ~=~ \A(t)\y(t) + \b(t),$$
kjer je $\A \in \glnr$, $\b \in \R^n$. 

\end{definicija}
\vspace{0.5cm}

\begin{definicija}

\textit{Začetni pogoj} (ali \textit{Cauchyjeva naloga}) za sistem $n$ linearnih diferenčnih enačb 1. reda je: reši 
$$\y(t+1) ~=~ \A(t)\y(t) + \b(t)$$
pri začetnem pogoju $\y(t_0) = \y_0 \in \R^n$.

\end{definicija}
\vspace{0.5cm}

\begin{izrek}

Prostor rešitev homogenega sistema je $n$-dimenzionalen vektorski prostor. Rešitve so linearno neodvisne v času $t+m$ $\iff$ so linearno neodvisne v času $t$.

\end{izrek}
\vspace{0.5cm}

\begin{komentar}[Nehomogen sistem]

Opazili smo, da če $\y,\z$ rešita
$$\y(t+1) ~=~ \A\y(t) + \b(t),$$
potem $\y(t) - \z(t) = \w(t)$ reši
$$\w(t+1) ~=~ \A\w(t).$$
Posledično je vsaka rešitev nehomogenega sistema oblike 
$$\y ~=~ \y_h + \y_p,$$
kjer je $\y_h$ rešitev homogenega sistema, $\y_p$ pa - pravimo ji \textit{partikularna} - rešitev nehomogenega sistema.

\end{komentar}
\vspace{0.5cm}

\begin{definicija}

Cauchyjeva naloga za linearne diferenčne enačbe s \textit{konstantnimi koeficienti} je: reši
$$y_{t+n} + a_{n-1}y_{t+n-1} + \ldots + a_0y_t ~=~ g(t)$$
pri pogoju
$$y(0) ~=~ \gamma_0,\ldots,y(n-1) ~=~ \gamma_{n-1}.$$
Enačbo lahko prevedemo na sistem:
$$\begin{bmatrix}
y_1(t+1) \\
y_2(t+1) \\
\vdots \\
y_{n-1}(t+1) \\
y_{n}(t+1)
\end{bmatrix} ~=~ \begin{bmatrix}
0 & 1 & 0 & \ldots & 0 \\
0 & 0 & 1 & \ldots & 0 \\
\vdots & \vdots & \vdots & \ddots & \vdots \\
0 & 0 & 0 & \ldots & 1 \\
-a_0 & -a_1 & -a_2 & \ldots & -a_{n-1} 
\end{bmatrix} \begin{bmatrix}
y_1(t) \\
y_2(t) \\
\vdots \\
y_{n-1}(t) \\
y_n(t)
\end{bmatrix} + \begin{bmatrix}
0 \\
0 \\
\vdots \\
0 \\
-g(t)
\end{bmatrix},$$
kjer je $y_k(t) := y(t+k-1)$ (torej velja $y_k(t) = y_{k-1}(t+1))$.

\end{definicija}
\vspace{0.5cm}

\begin{izrek}

Splošna rešitev homogene enačbe je dana z
$$y_H ~=~ \sum_{i=1}^n a_i y_i,$$
kjer so $y_i$, $i=1,\ldots,n$ linearno neodvisne rešitve enačbe. Prostor rešitev je vektorski prostor dimenzije $n.$

\end{izrek}
\vspace{0.5cm}

\begin{metoda}

Za iskanje linearnih neodvisnih rešitev homogene enačbe uporabimo nastavek $y_t = \lambda^t$:
$$\lambda^{t+n} + a_{n-1}\lambda^{t+n-1} + \ldots + a_0\lambda^t ~=~ 0$$
Karakteristični polinom:
$$p(\lambda) ~=~ \lambda^n + a_{n-1}\lambda^{n-1} + \ldots + a_n.$$
Naj bo $\lambda_0$ ničla $p(\lambda)$, $p(\lambda_0) = 0$. Potem je
$$\lambda_0^{t+n} + \ldots + a_0\lambda_0^t ~=~ \lambda_0^t \cdot p(\lambda_0) ~=~ 0.$$
Če ima $p(\lambda)$ $n$ enostavnih ničel, $\lambda_0,\ldots,\lambda_{n-1}$, je vsaka rešitev oblike
$$y(t) ~=~ \alpha_0\lambda^t + \ldots + \alpha_{n-1}\lambda_{n-1}^t.$$

\end{metoda}
\vspace{0.5cm}

% *************************************************************************************************

\subsection{Stabilnost}
\vspace{0.5cm}

\begin{definicija}

Naj bo $\y_{t+1} = \f(t,y_t)$ dan sistem in postavimo $t_0 = 0$. Rešitev $\y$ je \textit{stabilna}, če za $\forall \varepsilon > 0$ $\exists \delta > 0$: če je $\z$ katerakoli druga rešitev, ki zadošča $|\z_0 - \y_0| < \delta$ $\Rightarrow$ $|\z_t - \y_t| < \varepsilon$, $t>0$. \\

\noindent Rešitev je \textit{asimptotsko stabilna}, če je stabilna in za $\forall \varepsilon>0$ $\exists \delta>0$: \hbox{$|\z_0 - \y_0| < \delta$}, $|\z_t - \y_t| < \varepsilon$, $t\geq 0$ in $\lim_{t \rightarrow \infty} |\z_t - y_t| = 0$. \\

\noindent Za linearne sisteme:
\begin{align*}
\y_{t+1} ~&=~ \A \y_t + \b_t \\
\y_n ~&=~ \y_{n,P} + \A^n \y_0, ~~~y_{0,P} = 0 \\
\z_n ~&=~ \y_{n,P} + \A^n \z_0 \\
|\y_n - \z_n| ~&=~ |\A^n(\y_0 - \z_0)| ~\leq~ \|\A\|^n \|\y_0 - \z_0\|
\end{align*}
$\y_{n,P}$ je (asimptotsko) stabilna rešitev nehomogenega sistema $\iff$ $\mathbf{0}$ je (asimptotsko) stabilna rešitev homogenega sistema. Torej: $\|\A\| <1$: stabilnost.

\end{definicija}
\vspace{0.5cm}

%**************************************************************************************************

\pagebreak

% #################################################################################################

\section{Navadne diferencialne enačbe}
\vspace{0.5cm}

\begin{definicija}

\textit{Navadna diferencialna enačba} je vsaka enačba oblike
$$f(x,y(x),y'(x),\ldots,y^{(n)}(x)) ~=~ 0.$$
Red najvišjega odvoda je \textit{red} enačbe. \\

\noindent Cauchyjeva naloga za NDE $n$-tega reda je: reši
\begin{equation} \label{eq:1}
y^{(n)} ~=~ f(x,y,\ldots,y^{(n-1)})
\end{equation}
pri pogoju
$$(x_0,y(x_0),\ldots,y^{(n-1)}(x_0)) ~:=~ (x_0,y_0,\ldots,y_{n-1}) \in D_f.$$
Če $f$ ni eksplicitno odvisna od $x$, se enačba imenuje \textit{avtonomna}. Če ke $f$ linearna v $y,\ldots,y^{(n-1)}$, je enačba (\ref{eq:1}) linearna.

\end{definicija}
\vspace{0.5cm}

\begin{definicija}

Naj bo $D \subset \R^2$, odprta, $f: D \rightarrow \R$ dana funkcija in $(x_0,y_0) \in D$ začetni pogoj. Potem je funkcija $y$ rešitev
$y' ~=~ f(x,y), ~~~y(x_0) ~=~ y_0$,
na okolici $x_0$, če $\exists \delta > 0$:
$$y'(x) ~=~ f(x,y(x))$$
na $(x_0 - \delta, x_0 + \delta)$, $y(x_0) = y_0$.

\end{definicija}
\vspace{0.5cm}

\pagebreak

% #################################################################################################

\section{Linearne diferencialne enačbe}
\vspace{0.5cm}

% *************************************************************************************************

\subsection{LDE 1. reda}
\vspace{0.5cm}

\begin{enacba}

$$y' ~=~ f(x)y + g(x)$$
Naj bo $x_0$ izbrana točka. Potem je vsaka rešitev oblike
$$y ~=~ Ce^{\int_{x_0}^x f(t) \,dt} + y_p.$$
Prostor rešitve homogene enačbe $y' = f(x)y$ je $1$-dimenzionalni vektorski prostor.

\end{enacba}
\vspace{0.5cm}

% *************************************************************************************************

\subsection{Bernoullijeva DE}
\vspace{0.5cm}

\begin{enacba}

$$y' + p(x)y ~=~ q(x)y^{\alpha}.$$
Če $\alpha \in \set{0,1}$, potem je BDE kar LDE. Če ni, se v LDE prevede s substitucijo $z = y^{1-\alpha}$.

\end{enacba}
\vspace{0.5cm}

% *************************************************************************************************

\subsection{Ricattijeva DE}
\vspace{0.5cm}

\begin{enacba}

$$y' ~=~ a(x)y^2 + b(x)y + c(x), ~~~a,b,c \in \mathcal{C}([a,b])$$
V splošnem ni rešljiva. Če pa eno rešitev imamo, npr. $y_1$, s substitucijo $y = y_1 + z$ enačbo prevedemo na BDE za $z$.

\end{enacba}
\vspace{0.5cm}

% *************************************************************************************************

\subsection{LDE višjih redov s konstantnimi koeficienti}
\vspace{0.5cm}

\begin{definicija}

\textit{Nehomogena LDE $n$-tega reda s konstantnimi koeficienti} je enačba oblike
$$L(y) ~:=~ y^{(n)} + a_1 y^{(n-1)} + \ldots + a_n y ~=~ f(x).$$
Če $f(x) \equiv 0$ je enačba homogena.

\end{definicija}
\vspace{0.5cm}

\begin{trditev}

Prostor rešitev enačbe $L(y) = 0$ je $n$-dimenzionalen vektorski prostor. Če je $y_p$ rešitev enačbe
$$L(y_p) ~=~ f,$$
potem je vsaka druga rešitev te enačbe oblike
$$y ~=~ y_h + y_p,$$
kjer $L(y_h) = 0$.

\end{trditev}
\vspace{0.5cm}

% *************************************************************************************************

\pagebreak

% #################################################################################################

\section{Eksistenca, enoličnost, gladkost}
\vspace{0.5cm}

% *************************************************************************************************

\subsection{Uvod}
\vspace{0.5cm}

\begin{izrek}[Lokalni eksistenčni izrek]

Naj ob $y' = f(x,y)$ diferencialna enačba, $f \in \C((0,a)\times(0,b))$, $f$ Lipschitzova na 2. spremenljivko pri fiksni prvi spremenljivki $x$ s koeficientom $k(x)$, ki je lokalno integrabilna na $(0,a)$. Potem lahko za $\forall (x_0,y_0) \in (0,a) \times (0,b)$ obstaja natanko ena rešitev enačbe pri začetnem pogoju $y(x_0) = y_0$, ki je definirana na neki okolici $x_0$. 

\end{izrek}
\vspace{0.5cm}

\begin{izrek}[Globalni eksistenčni izrek]

Naj bo $y' = f(x,y)$ diferencialna enačba, $f \in \C([0,a] \times \R)$, Lipschitzova na 2. spremenljivko pri fiksni 1. spremenljivki s konstanto $k(x)$, ki je integrabilna na $[0,a]$. Potem obstaja natanko ena rešitev enačba pri pogoju $y(0) = y_0$, ki je definirana na $[0,a]$.

\end{izrek}
\vspace{0.5cm}

% *************************************************************************************************

\subsection{Gladkost}
\vspace{0.5cm}

\begin{izrek}

Cauchyjeva naloga:
\begin{align*}
y' ~&=~ f(x,y,a), ~~~a \in P \subset \R^n, \\
y(s) ~&=~ t
\end{align*}
$\varphi(x,x_0,y_0)$ je rešitev zgornje Cauchyjeve naloge.
Naj bo $f$ gladka v vseh spremenljivkah. Potem je $\varphi$ gladko odvisna od $x,s,t,a$.

\end{izrek}
\vspace{0.5cm}

% *************************************************************************************************

\pagebreak

% #################################################################################################

\section{Sistemi diferencialnih enačb}
\vspace{0.5cm}

% *************************************************************************************************

\subsection{Uvod}
\vspace{0.5cm}

\begin{definicija}

Naj bodo $x,y_1,\ldots,y_n$ realne spremenljivke in $f_i: \R^{n+1} \rightarrow \R$, $i = 1,\ldots,n$, realna funkcija. Sistem $n$ diferencialnih enačb 1. reda je
\begin{align*}
y'_1 ~&=~ f_1(x,y_1,\ldots,y_n) \\
&\vdots \\
y'_n ~&=~ f_n(x,y_1,\ldots,y_n).
\end{align*}
V vektorski obliki:
$$\y ~=~ \begin{bmatrix}
y_1 \\
\vdots \\
y_n
\end{bmatrix}, ~~~\f ~=~ \begin{bmatrix}
f_1 \\
\vdots \\
f_n
\end{bmatrix}.$$
Sistem se prepiše v $\y' = \f(x,\y)$. Cauchyjeva naloga:
\begin{align*}
\y' ~&=~ \f(x,\y) \\
\y(x_0) ~&=~ \y_0
\end{align*}
Če $x$ v $\f$ eksplicitno ne nastopa, se sistem imenuje \textit{avtonomen}. Če je $\f$ linearna v $\y$, je sistem \textit{linearen} in pišemo
$$\y' ~=~ \A(x)\y + \b(x).$$
Če je $\b \equiv 0$, je sistem \textit{homogen}. 

\end{definicija}
\vspace{0.5cm}

% *************************************************************************************************

\subsection{Sistemi linearnih diferencialnih enačb}
\vspace{0.5cm}

\begin{posledica}

Sistem
$$\y' ~=~ \A(x)\y + \b(x),$$
$\A,\b$ zvezni na $[0,a]$, izpolnjuje predpostavke G.E.I.

\end{posledica}
\vspace{0.5cm}

\begin{izrek}

Naj bo 
$$\y' ~=~ \A(x)\y + \b(x)$$
sistem $n$ diferencialnih enačb 1. reda, $\A,\b$ zvezni na $[0,a]$. Prostor rešitev homogene enačbe $\y = \A(x)\y$ je vektorski prostor dimenzije $n$. Vsaka rešitev nehomogenega je oblike
$$\y ~=~ \y_H + \y_P,$$
kjer je $\y_H$ rešitev homogenega sistema, $\y_P$ pa partikularna rešitev nehomogenega sistema.

\end{izrek}
\vspace{0.5cm}

% *************************************************************************************************

\pagebreak

% #################################################################################################

\section{Kvalitativna analiza}
\vspace{0.5cm}

% *************************************************************************************************

\subsection{Uvod}
\vspace{0.5cm}

\begin{definicija}

Naj bo $\y' = \f(x,\y)$ dan sistem enačb in $\bm{\varphi}(x)$ njegova rešitev. Potem pravimo, da je $\bm{\varphi}$ \textit{stabilna}, da za $\forall \varepsilon>0  ~\exists \delta > 0$, da velja: če je $\bm{\psi}$ neka druga rešitev in je $\|\bm{\psi}(0) - \bm{\varphi}(0)\| < \delta$, potem
$$\|\bm{\psi}(x) - \bm{\varphi}(x)\| ~<~ \varepsilon, \quad x > 0.$$
Če je $\bm{\varphi}$ stabilna in za $\bm{\psi}$, ki zadošča zgornjim pogojem, velja še
$$\lim_{x \rightarrow \infty} \|\bm{\psi}(x) - \bm{\varphi}(x)\| ~=~ 0,$$
pravimo, da je $\bm{\varphi}$ \textit{asimptotsko stabilna}.

\end{definicija}
\vspace{0.5cm}

\begin{izrek}

Naj bo $\y' = \A\y$ dan sistem in je $\A$ matrika z lastnimi \hbox{vrednostmi $\lambda_i$.}
\begin{enumerate}
	
\item[(a)] Če $\forall i:\re(\lambda_i) < 0$ je vsaka rešitev asimptotsko stabilna.

\item[(b)] Če $\exists i: \re(\lambda_i) > 0$ so vse rešitve nestabilne.

\item[(c)] Če $i<j: \re(\lambda_i) = 0$ in je število lastnih vektorjev, ki pripadajo $\lambda_i$, enako njeni algebrajski večkratnosti in $i \geq j: \re(\lambda_i) < 0$, je vsaka rešitev stabilna. Če $\exists i<j$, tako da algebrajska večkratnost ni enaka številu lastnih vektorjev ($\re(\lambda_i)=0$), potem je vsaka rešitev nestabilna.	
	
\end{enumerate}

\end{izrek}
\vspace{0.5cm}

\begin{definicija}

Naj bo $\y' = \A\y$ dan sistem in naj bo $\vv_j = \uu_j + i\w_j$ posplošeni lastni vektor (lastni ali korenski) za lastno vrednost $\lambda_j$. Definiramo:
\begin{itemize}

\item \textit{Stabilni podprostor}:
$$E^s ~=~ \Lin\set{\uu_j, \w_j \in \R^n \mid \re(\lambda_j)<0}$$

\item \textit{Nestabilni podprostor}:
$$E^n ~=~ \Lin\set{\uu_j, \w_j \in \R^n \mid \re(\lambda_j)>0}$$

\item \textit{Centralni podprostor}:
$$E^c ~=~ \Lin\set{\uu_j, \w_j \in \R^n \mid \re(\lambda_j)=0}$$

\end{itemize}

\end{definicija}
\vspace{0.5cm}

% *************************************************************************************************

\subsection{Stabilnost ravnovesnih rešitev}
\vspace{0.5cm}

\begin{definicija}

Stacionarna točka $\y_0$ avtonomnega sistema $\y' = \f(\y)$, $\f$ gladka, je \textit{hiperbolična}, če ima
$$\A ~=~ \mathbf{D}\f(\y)$$
trivialen centralni prostor. Če $\y_0$ ni hiperbolična, je \textit{nehiperbolična}.

\end{definicija}
\vspace{0.5cm}

\begin{izrek}[Harman-Grobman]

Naj bo $E \subset \R^n$ odprta množica, ki vsebuje izhodišče in naj bo $\f: E \rightarrow \R^n$ razreda $\C^2$. Označimo z $\bm{\varphi}_x(t)$ rešitev, ki zadošča $\bm{\varphi}_x(0) = x$ (tokovnica skozi $x$). Naj bo $\0$ hiperbolična stacionarna točka sistema in pišimo
$$\A ~=~ \mathbf{D}\f(\0).$$
Obstaja $H: U \rightarrow V$, difeomorfizem reda $\C^1$, $U,V$ odprti okolici izhodišča, $H(0)=(0)$ tak, da velja: $\forall x_0 \in U$ obstaja odprt interval $I_0$ okoli $0$, $I_0 \subset \R$ in velja
$$H \circ \bm{\varphi}_{x_0} ~=~ e^{\A t} H(x_0).$$

\end{izrek}
\vspace{0.5cm}

\begin{opomba}

Če je $\f$ zgolj razreda $\C^1$, je $H$ homeomorfizem.

\end{opomba}
\vspace{0.5cm}

\begin{posledica}

Ob predpostavkah izreka Hartman-Grobman je $x=0$ asimptotsko stabilna rešitev linearizirane enačbe $\x' = \A\x$. Rešitev je nestabilna, če ima $\A$ vsaj eno lastno vrednost $\lambda$, $\re(\lambda)>0$ ($\iff E^n \neq \set{0})$.

\end{posledica}
\vspace{0.5cm}

% *************************************************************************************************

\subsection{Problem morskih psov med 1. svetovno vojno}
\vspace{0.5cm}

\begin{lema}

Naj bosta $x(t),y(t)$ periodični rešitvi sistem Lotka-Volterra s periodo $T>0$. Potem je
\begin{align*}
\frac{l}{b} ~&=~ \frac{1}{T}\int_0^T x(t)\,dt, \\
\frac{k}{a} ~&=~ \frac{1}{T}\int_0^T y(t)\,dt.
\end{align*}

\end{lema}
\vspace{0.5cm}

% *************************************************************************************************

\subsection{Limitne rešitve}
\vspace{0.5cm}

\begin{izrek}

Naj bo $\y' = \f(\y)$ dan sistem, $\f\in\C^1$, $\varphi(t)$ rešitev in \hbox{$\lim_{t \rightarrow \infty} \varphi(t) = \y_0$}. Potem je $\y_0$ stacionarna točka.

\end{izrek}
\vspace{0.5cm}

% *************************************************************************************************

\subsection{Stabilnost in funkcija Ljapunova}
\vspace{0.5cm}

\begin{izrek}

Naj bo $E \subset \R^2$ odprta množica, ki vsebuje izhodišče. Naj bo $\dot{\x}=\f(\x)$ dan sistem in naj bo $\0$ njegova stacionarna točka in naj obstaja gladka vunkcija $V$ na $E$, $V(0)=0$, $V>0$ sicer.
\begin{enumerate}
	\item[(a)] $\dot{V} \leq 0$: $\0$ je stabilna;
	\item[(b)] $\dot{V}<0$ na $E \setminus \set{\0}$: $\0$ je asimptotsko stabilna;
	\item[(c)] $\dot{V}>0$ na $E \setminus \set{\0}$: $\0$ je nestabilna. 
\end{enumerate}
Funkcija $V$ se imenuje \textit{Ljupanova}.

\end{izrek}
\vspace{0.5cm}

% *************************************************************************************************

\subsection{Vedenje pri velikih časih}
\vspace{0.5cm}

\begin{izrek}[Poincare-Bendixon]

Naj bo $\y' = \f(\y)$ dan sistem, $K \subset \R^2$ kompaktna množica brez stacionarnih točk sistemu in naj bo $\varphi_x(t)$, $t \geq 0$, točkovnica, ki je v $K$ za $\forall t \geq 0$:
$$\varphi_x(t) \in K, \quad \forall t \geq 0.$$
Potem obstaja v $K$ periodična rešitev in $\varphi_x(t)$ se navija okoli nje.

\end{izrek}
\vspace{0.5cm}

% *************************************************************************************************

\pagebreak

% #################################################################################################

\end{document}