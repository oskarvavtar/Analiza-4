\documentclass[11pt]{article}
\usepackage[utf8]{inputenc}
\usepackage[slovene]{babel}

\usepackage{amsthm}
\usepackage{amsmath, amssymb, amsfonts}
\usepackage{relsize}
\usepackage{mathrsfs}

\newcommand{\R}{\mathbb{R}}
\newcommand{\N}{\mathbb{N}}
\newcommand{\dif}{\Delta}
\renewcommand{\b}{\mathbf{b}}
\newcommand{\f}{\mathbf{f}}
\newcommand{\y}{\mathbf{y}}
\newcommand{\A}{\mathbf{A}}

\theoremstyle{definition}
\newtheorem{definicija}{Definicija}[section]

\theoremstyle{definition}
\newtheorem{trditev}{Trditev}[section]

\theoremstyle{definition}
\newtheorem{izrek}{Izrek}[section]

\theoremstyle{definition}
\newtheorem{metoda}{Metoda}[section]

\newtheorem*{posledica}{Posledica}
\newtheorem*{opomba}{Opomba}
\newtheorem*{komentar}{Komentar}
\newtheorem{lema}{Lema}
\newtheorem*{dokaz}{Dokaz}
\newtheorem*{posplošitev}{Posplošitev}
\newtheorem*{dogovor}{Dogovor}
\newtheorem*{sklep}{Sklep}

\title{Analiza 4 - definicije, trditve in izreki}
\author{Oskar Vavtar \\
po predavanjih profesorice Jasne Prezelj}
\date{2019/20}

\begin{document}
\maketitle
\pagebreak
\tableofcontents
\pagebreak

% #################################################################################################

\section{Diferenčne enačbe}
\vspace{0.5cm}

% *************************************************************************************************

\subsection{Uvod}
\vspace{0.5cm}

\begin{definicija}[Diferenca]

Denimo, da je $y=f(t)$ dana funkcija.
\begin{itemize}
	\item 1. način: $\dif y_t ~=~ f(t+h) - f(t) ~=~ y_{t+h} - y_t$
	\item 2. način: $\dif y_t ~=~ f(t) - f(t-h) ~=~ y_t - y_{t-h}$
\end{itemize}
Posebej definiramo $\dif^0 y_t = y_t$. Velja
$$\dif^{n+1} y_t ~=~ \dif(\dif^n y_t)_t.$$

\end{definicija}
\vspace{0.5cm}

\begin{definicija}

\textit{Navadna diferenčna enačba} je enačba, ki vsebuje (eno ali) več diferenc,
$$F(t, \dif^0 y_t, \ldots, \dif^n y_t) ~=~ 0.$$
\textit{Red} diferenčne enačbe je red najvišje diference. Če je $F$ linearna v $\dif^k y_t$, je enačba linearna.

\end{definicija}
\vspace{0.5cm}

\begin{definicija}

\textit{Sistem $n$ diferenčnih enačb 1. reda} je dan z
\begin{align*}
y_1(t+1) ~&=~ f_1(t,y_1(t),\ldots,y_n(t)) \\
&\vdots \\
y_n(t+1) ~&=~ f_n(t,y_1(t),\ldots,y_n(t))
\end{align*} 
Če $t$ eksplicitno ne nastopa, rečemo, da je to \textit{avtonomen sistem}. Če so $f_1,\ldots,f_n$ linearne, lahko sistem zapišemo v matrični obliki:
\begin{align*}
\y ~&=~ \begin{bmatrix}
y_1 \\
\vdots \\
y_n
\end{bmatrix},~~~ f_i(t) ~=~ b_i(t) + a_{i1}y_1(t) + \ldots + a_{in}y_n(t), ~i = 1,\ldots, n\\ \\
\y(t+1) ~&=~ \begin{bmatrix}
a_{11}(t) & \ldots & a_{1n}(t) \\
\vdots & \ddots & \vdots \\
a_{n1}(t) & \vdots & a_{nn}(t)
\end{bmatrix} \begin{bmatrix}
y_1(t) \\
\vdots \\
y_n(t) 
\end{bmatrix} + \begin{bmatrix}
b_1(t) \\
\vdots \\
b_n(t)
\end{bmatrix} ~=~ \A(t)\y(t) + \b(t)
\end{align*}
Nelinearen sistem lahko vseeno zapišemo v vektorski obliki:
$$\f ~=~ \begin{bmatrix}
f_1 \\
\vdots \\
f_n
\end{bmatrix}, ~~~ \y(t+1) ~=~ \f(t, \y(t))$$

\end{definicija}
\vspace{0.5cm}

\begin{definicija}

Če je sistem oblike
$$\y_{m+1} = \A\y_m,$$
se imenuje \textit{homogen}.

\end{definicija}
\vspace{0.5cm}

\pagebreak

% *************************************************************************************************

% #################################################################################################

\end{document}